

%\documentclass[12pt,a4paper]{book}
\documentclass[12pt,a4paper]{book}

\usepackage{listings}
\usepackage{color}

\definecolor{mygreen}{rgb}{0,0.6,0}
\definecolor{mygray}{rgb}{0.5,0.5,0.5}
\definecolor{mymauve}{rgb}{0.58,0,0.82}

\lstset{ %
  backgroundcolor=\color{white},   % choose the background color; you must add \usepackage{color} or \usepackage{xcolor}
  basicstyle=\scriptsize,        % the size of the fonts that are used for the code
  breakatwhitespace=false,         % sets if automatic breaks should only happen at whitespace
  breaklines=true,                 % sets automatic line breaking
  captionpos=b,                    % sets the caption-position to bottom
  commentstyle=\color{mygreen},    % comment style
  deletekeywords={unsigned, int, float},            % if you want to delete keywords from the given language
  escapeinside={\%*}{*)},          % if you want to add LaTeX within your code
  extendedchars=true,              % lets  you use non-ASCII characters; for 8-bits encodings only, does not work with UTF-8
  frame=single,                    % adds a frame around the code
  keepspaces=true,                 % keeps spaces in text, useful for keeping indentation of code (possibly needs columns=flexible)
%  keywordstyle={},
  keywordstyle=\color{blue},       % keyword style
  language=C++,                 % the language of the code
  keywords={cudaEvent_t},
  %morekeywords={*,...,cudaError_t, cudaStream_t, max},            % if you want to add more keywords to the set
  numbers=left,                    % where to put the line-numbers; possible values are (none, left, right)
  numbersep=5pt,                   % how far the line-numbers are from the code
  numberstyle=\tiny\color{mygray}, % the style that is used for the line-numbers
  rulecolor=\color{black},         % if not set, the frame-color may be changed on line-breaks within not-black text (e.g. comments (green here))
  showspaces=false,                % show spaces everywhere adding particular underscores; it overrides 'showstringspaces'
  showstringspaces=false,          % underline spaces within strings only
  showtabs=false,                  % show tabs within strings adding particular underscores
  stepnumber=2,                    % the step between two line-numbers. If it's 1, each line will be numbered
  stringstyle=\color{mymauve},     % string literal style
  tabsize=2,                      % sets default tabsize to 2 spaces
%  title=\lstname                   % show the filename of files included with \lstinputlisting; also try caption instead of title
}

\newcommand\myvspace{\vspace{\baselineskip}}
\usepackage[CJKbookmarks, colorlinks,
bookmarksnumbered=true,
pdfstartview=FitH,
linkcolor=black]{hyperref}

%\usepackage{geometry} % 設定邊界
%\geometry{
 % top=1in,
  %inner=1in,
  %outer=1in,
  %bottom=1in,
  %headheight=3ex,
  %headsep=2ex
%}
% fc-list : lang=zh
\usepackage{fontspec} % 允許設定字體
\usepackage{xeCJK} % 分開設置中英文字型
%\setCJKmainfont{LiHei Pro} % NOT OK
\setCJKmainfont{STFangsong}  % OK
%\setCJKmainfont{Yuanti SC} %OK
%\setCJKmainfont{STHeiti} %OK
%\setCJKmainfont{LiSong Pro} % NOT OK
%\setCJKmainfont{STKaiti} %OK
%\setCJKmainfont{Kai} %OK
%\setCJKmainfont{Kaiti TC} % Traditional Chinese OK!
%\setCJKmainfont{Wawati SC} % 娃娃体
%\setCJKmainfont{Lantinghei SC} %WORKING
%\setCJKmainfont{Libian SC} %WORKING
%\setCJKmainfont{Source Han Sans CN ExtraLight} % NOT WORKING

%set jusha source code base
\newcommand{\JushaBase}{../}
%
\setmainfont{Georgia} % 設定英文字型
\setromanfont{Georgia} % 字型
\setmonofont{Courier New}
\linespread{1.2}\selectfont % 行距
\XeTeXlinebreaklocale "zh" % 針對中文自動換行
\XeTeXlinebreakskip = 0pt plus 1pt % 字與字之間加入0pt至1pt的間距,確保左右對整齊
\parindent 0em % 段落縮進
\setlength{\parskip}{20pt} % 段落之間的距離
 
% yp: this works for simplified chinese
\setCJKfamilyfont{pmingliu}{PMingLiU} % 設定新字型(新細明體)
\newcommand\fontml{\CJKfamily{pmingliu}} % 新增指令\fontml應用字型

% in mac list the fonts: fc-list 
% this does not
\setCJKfamilyfont{biaukai}{BiauKai} % 設定新字型(標階體)
\newcommand\fontbk{\CJKfamily{biaukai}} % 新增指令\fontbk應用字型
 
\title{{\huge 聚沙成塔} \\ 我的CUDA笔记}% 設置標題,使用巨大字體

\author{张永鹏} % 設置作者
%\date{February 2013} % 設置日期
\usepackage{titling}
\setlength{\droptitle}{-8em} % 將標題移動至頁面的上面
 
\usepackage{fancyhdr}
\usepackage{lastpage}
\pagestyle{fancyplain}
\lhead{\fancyplain{}{code.accelerate@gmail.com}} % 左頁首
\chead{} % 中頁首
\rhead{\fancyplain{}{我的CUDA笔记}} % 右頁首
\lfoot{} % 左頁尾
\cfoot{\fancyplain{}{\thepage\ of \pageref{LastPage}}} % 中頁尾
\rfoot{} % 右頁尾
 
\begin{document}
 
\clearpage
 
\maketitle % 顯示標題

\begin{center}
\vspace*{4in}
\large{
“乃至童子戏,聚沙为佛塔。如是诸人等,皆已成佛道。”\\
            \par\par\par\par\par\par \par\par\par\par                          ------《妙法莲华经·方便品》
}
\end{center}

%\tableofcontents
% \input{xelatex_ex.tex}
% !TEX program = xelatex
% !TEX root = xelatex_main.tex
% !TEX encoding = UTF-8 Unicode


\chapter{什么是CUDA}

CUDA是一个并行计算的编程模型(programming model)。它是由NVIDIA在2007年提出的一种针对让自己的图形加速卡用于通用计算的一个突破。

进入21世纪初以来,芯片的并行性越来越高,这并不是由于。在这之前通用计算的芯片的性能提高主要是由时钟频率的提示带来的。
% !TEX root = xelatex_main.tex
% !TEX encoding = UTF-8 Unicode
\chapter{基本工具}

这章主要介绍一些个人随身携带的工具。 编程语言是C++/CUDA,并且尽可能的使用C++11标准的一些新特性。CUDA推荐6.0以上,而且越新越好。代码基本在LINUX下编译,不追求跨平台(换句话说WINDOWS拜拜了)。所有自写的代码都在jusha的namespace下,以避免和其它库冲突。

\section{测量时间}
我们使用CUDA的(一个)最主要原因就是对性能的高要求,所以一个好的测量时间的工具是必不可少的。在LINUX下比较常用的是gettimeofday(),WINDOWS一般用QueryPerformanceFrequency 和 QueryPerformanceCounter。这些对测量CPU的时间是比较合适的。但在CUDA里,由于CUDA内核的调用对CPU来说是异步的,如果使用上述方法的话必须非常小心,即在测量开始和结束的时候必须使用cudaDeviceSynchronize()等来保证CPU和GPU的同步,否则就会出现测量得到的时间比实际的时间要短很多的情况。另外一个方法是设置环境变量CUDA\_LAUNCH\_BLOCKING为1(缺省为0),这样的话每一个内核的调用都变成了同步调用,也就是说CPU会在内核执行完之后才返回。但这样做的缺点是程序性能会下降,因为原本CPU可以提前返回做其它的事情。所幸的是CUDA提供了一套基于event的API来帮助测量时间,内部用GPU自带的计数器,所以不占用太多CPU资源。相关的接口函数如下所列:
\myvspace
\lstinputlisting[language=C++]{code/event_api.cpp}

从函数名上就基本上可以猜出怎么使用了,一小段范例如下:
\myvspace
\lstinputlisting[language=C++]{code/event_example.cpp}
 
基于这套API我们就可以简单的构造一个类把这些API包裹起来,然后定义几个函数来方便调用,这样每次加入测量就不用查手册了。


\section{thrust库/头文件}

要用CUDA就不得不提到CUDA自带的thrust库。其实准确的说thrust并不是一个预编译的库,而是一些模版的头文件。
在cuda 6.0 thrust的原程序里还可以找到像这样的注释:

\begin{lstlisting}
 // N.B. -- I wish we could do some pragma unrolling here too, but the compiler makes it 1% slower
\end{lstlisting}
 
\subsection{试写一个bitmap scan}

在实际应用中有很多情况下我们想要,这样的话似乎
首先我们先来熟悉一下什么是scan。也叫prefix sum。
它的C语言的实现可以这样:

\myvspace
\lstinputlisting[language=C++]{\JushaBase/src/x86/detail/exclusive_scan_v0.h}

当然如果想要支持其它的类型,那么C++的模版可以派上用场了。
\myvspace
\lstinputlisting[language=C++]{\JushaBase/src/x86/detail/exclusive_scan_v1.h}
如果还想更灵活一点,比如实现min/max的scan的话,我们还可以加上另外一个模版变量Op,
\myvspace
\lstinputlisting[language=C++]{\JushaBase/src/x86/detail/exclusive_scan_v2.h}

作为C++程序员,写到这里就可以交差了。就算想再用多线程加速,估计提高也不大,因为这个函数是被带宽限制了。

而这只是CUDA程序员的万里长征的开始而已。

首先要想想怎么并行化,因为这是一个看上去对非常顺序的操作,每一个输出的结果都和前一个输出相关。
%有了这个类,可以使用两个函数(一个开始,一个结束)把任何一段需要测时间的代码包含进来,并且用一个string来区分各个。

%\begin{figure}[htbp] %  figure placement: here, top, bottom, or page
%   \centering
%   \includegraphics[width=2in]{example.pdf} 
%   \caption{example caption}
%   \label{fig:example}
%\end{figure}
%

% !TEX root = xelatex_main.tex
\chapter{理解内存合并访问}

有很多内核里浮点或整形运算量不能完全掩盖内存访问延迟,也就是说它们的性能是被内存的带宽限制的(memory bounded)。 对于这些内核来说,理解并尽可能地满足CUDA里的内存合并访问(Coalesced Memory Access)是提升内核性能的最主要手段之一。 让我们从一个简单的例子来看看到底怎么样才能最大限度地利用GPU上的巨大带宽。这个例子的任务是用内核函数来实现两块GPU内存之间的拷贝。从功能上来说这是CUDA提供的cudaMemcpy的一种(cudaMemcpyDeviceToDevice),貌似没有什么必要,但是我们可以和这个API做一下性能上的比较,看看我们的实现是否和CUDA提供的cudaMemcpy相仿。同时也能理解一下保证合并访问的条件。

作为格调更高的C++程序员,我们当然情不自禁地从模版开始码字了:template <typename T> ...,第一版本如下:
\myvspace
\lstinputlisting[language=C++]{\JushaBase/tests/src/d2d_direct.cu}
代码很简单,无非就是一个循环(第3,4行)把输入拷到输出。内核调用的时候用了一下jusha里提供的缺省的blockDim和gridDim(第9,10行)。
如果直接用CUDA提供的cudaMemcpy的话,就一句,更简单:
\myvspace
\lstinputlisting[language=C++]{\JushaBase/tests/src/cudamemcpy.cu}

首先我们来比较一下性能,这里画出不同数据大小在各种数据类型的测量带宽曲线,类型有char, int/float, double, float3, float4。

可以看到cudaMemcpy性能非常稳定,不管是任何类型,在  处都能达到理论带宽的%。而我们写的函数则优劣不齐。好的(float4, int, double)能接近cudaMemcpy,差的只能达到二分之一(char) 或者(flaot3)左右。

现在回到CUDA文档里合并访问的要求,如下:
\begin{itemize}
\item 对于每16个线程(半个warp)来说,
\end{itemize}

费了很大的周折,我们终于实现了一个无论数据类型,无论地址偏移的高性能的拷贝,归根结底还是要回到用C语言的实现。
 
\end{document}
