% !TEX program = xelatex
% !TEX root = xelatex_main.tex
% !TEX encoding = UTF-8 Unicode

\chapter{基本工具}

这章主要介绍一些个人随身携带的工具。 编程语言是C++/CUDA,并且尽可能的使用C++11标准的一些新特性。CUDA推荐6.0以上,而且越新越好。代码基本在LINUX/MAC OS X下编译,不追求跨平台(换句话说WINDOWS拜拜了)。所有自写的代码都在jusha的namespace下,以避免和其它库冲突。

\section{测量时间}
我们使用CUDA的(一个)最主要原因就是对性能的高要求,所以一个好的测量时间的工具是必不可少的。在LINUX下比较常用的是gettimeofday(),WINDOWS一般用QueryPerformanceFrequency 和 QueryPerformanceCounter。这些对测量CPU的时间是比较合适的。但在CUDA里,由于CUDA内核的调用对CPU来说是异步的,如果使用上述方法的话必须非常小心,即在测量开始和结束的时候必须使用cudaDeviceSynchronize()等来保证CPU和GPU的同步,否则就会出现测量得到的时间比实际的时间要短很多的情况。另外一个方法是设置环境变量CUDA\_LAUNCH\_BLOCKING为1(缺省为0),这样的话每一个内核的调用都变成了同步调用,也就是说CPU会在内核执行完之后才返回。但这样做的缺点是程序性能会下降,因为原本CPU可以提前返回做其它的事情。所幸的是CUDA提供了一套基于event的API来帮助测量时间,内部用GPU自带的计数器,所以不占用太多CPU资源。相关的接口函数如下所列:
\myvspace
\lstinputlisting[language=C++]{code/event_api.cpp}

从函数名上就基本上可以猜出怎么使用了,一小段范例如下:
\myvspace
\lstinputlisting[language=C++]{code/event_example.cpp}
 
基于这套API我们就可以简单的构造一个类把这些API包裹起来,然后定义几个函数来方便调用,这样每次加入测量就不用查手册了。

\section{项目配置}
对于项目配置,个人推荐CMake,因为这实在是跨平台的利器。不管是生成Makefile系统,还是Visual Studio的项目文件,还是Xcode,只需要写一个通用脚本就能通吃。以下提供一个最简单的生成带有CUDA文件的C++项目的CMake脚本。短短20行,省去了多少在Visual Studio下的鼠标操作啊。
\myvspace
\lstinputlisting{code/CMakelists.txt}

\section{thrust库/头文件}

要用CUDA就不得不提到CUDA自带的thrust库。其实准确的说thrust并不是一个预编译的库,而是一些模版的头文件。
在cuda 6.0 thrust的原程序里还可以找到像这样的注释:

\begin{lstlisting}
 // N.B. -- I wish we could do some pragma unrolling here too, but the compiler makes it 1% slower
\end{lstlisting}
 
\subsection{试写一个bitmap scan}

在实际应用中有很多情况下我们想要,这样的话似乎
首先我们先来熟悉一下什么是scan。也叫prefix sum。
它的C语言的实现可以这样:

\myvspace
\lstinputlisting[language=C++]{\JushaBase/src/x86/detail/exclusive_scan_v0.h}

当然如果想要支持其它的类型,那么C++的模版可以派上用场了。
\myvspace
\lstinputlisting[language=C++]{\JushaBase/src/x86/detail/exclusive_scan_v1.h}
如果还想更灵活一点,比如实现min/max的scan的话,我们还可以加上另外一个模版变量Op,
\myvspace
\lstinputlisting[language=C++]{\JushaBase/src/x86/detail/exclusive_scan_v2.h}

作为C++程序员,写到这里就可以交差了。就算想再用多线程加速,估计提高也不大,因为这个函数是被带宽限制了。

而这只是CUDA程序员的万里长征的开始而已。

首先要想想怎么并行化,因为这是一个看上去对非常顺序的操作,每一个输出的结果都和前一个输出相关。
%有了这个类,可以使用两个函数(一个开始,一个结束)把任何一段需要测时间的代码包含进来,并且用一个string来区分各个。

%\begin{figure}[htbp] %  figure placement: here, top, bottom, or page
%   \centering
%   \includegraphics[width=2in]{example.pdf} 
%   \caption{example caption}
%   \label{fig:example}
%\end{figure}
%
