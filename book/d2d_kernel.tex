% !TEX root = xelatex_main.tex
\chapter{优化一个简单内核}

让我们从一个简单的例子来看看优化CUDA有什么技巧。这个例子的任务是用内核函数来实现两块GPU内存之间的拷贝。从功能上来说这是CUDA提供的cudaMemcpy的一种(cudaMemcpyDeviceToDevice),貌似没有什么必要,但是我们可以和这个API做一下性能上的比较,看看我们的实现是否能比CUDA提供的功能快。同时也能理解一下使用coalesced memory access的要求。

作为格调更高的C++程序员,我们当然情不自禁地从模版开始码字了:template <typename T> ...,第一版本如下:
\myvspace
\lstinputlisting[language=C++]{\JushaBase/tests/src/d2d_direct.cu}
代码很简单,无非就是一个循环(第3,4行)把输入拷到输出。内核调用的时候用了一下jusha里提供的缺省的blockDim和gridDim(第9,10行)。
如果直接用CUDA提供的cudaMemcpy的话,就一句,更简单:
\myvspace
\lstinputlisting[language=C++]{\JushaBase/tests/src/cudamemcpy.cu}

首先我们来比较一下性能,这里画出不同数据大小在各种数据类型的测量带宽曲线,类型有char, int/float, double, float3, float4。

可以看到cudaMemcpy性能非常稳定,不管是任何类型,在  处都能达到理论带宽的%。而我们写的函数则优劣不齐。好的数据