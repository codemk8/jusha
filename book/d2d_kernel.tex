% !TEX program = xelatex
% !TEX root = xelatex_main.tex
% !TEX encoding = UTF-8 Unicode

\chapter{理解内存合并访问}

有很多内核里浮点或整形运算量不能完全掩盖内存访问延迟,也就是说它们的性能是被内存的带宽限制的(memory bounded)。 对于这些内核来说,理解并尽可能地满足CUDA里的内存合并访问(Coalesced Memory Access)是提升内核性能的最主要手段之一。 让我们从一个简单的例子来看看到底怎么样才能最大限度地利用GPU上的巨大带宽。这个例子的任务是用内核函数来实现两块GPU内存之间的拷贝。从功能上来说这是CUDA提供的cudaMemcpy的一种(cudaMemcpyDeviceToDevice),貌似没有什么必要,但是我们可以和这个API做一下性能上的比较,看看我们的实现是否和CUDA提供的cudaMemcpy相仿。同时也能理解一下保证合并访问的条件。

作为格调更高的C++程序员,我们当然情不自禁地从模版开始码字了:template <typename T> ...,第一版本如下:
\myvspace
\lstinputlisting[language=C++]{\JushaBase/tests/src/d2d_direct.cu}
代码很简单,无非就是一个循环(第3,4行)把输入拷到输出。内核调用的时候用了一下jusha里提供的缺省的blockDim和gridDim(第9,10行)。
如果直接用CUDA提供的cudaMemcpy的话,就一句,更简单:
\myvspace
\lstinputlisting[language=C++]{\JushaBase/tests/src/cudamemcpy.cu}

首先我们来比较一下性能,图\ref{fig:d2d_direct} 画出d2d\_direct\_kernel在不同数据大小和不同数据类型的带宽曲线,类型有char, int/float, double, float3, float4。对于cudaMemcpy, 由于它不区分数据类型,我们只画出一条曲线。另外,测试是在NVIDIA K40上,没有ECC。
\begin{figure}[h!] %  figure placement: here, top, bottom, or page
   \centering
   \includegraphics[width=\linewidth]{scripts/d2d_kernel/cudaMemcpy_vs_d2d.eps} 
   \caption{example caption}
   \label{fig:d2d_direct}
\end{figure}
%

可以看到cudaMemcpy性能非常稳定,在  处都能达到理论带宽的%。而我们写的函数则优劣不齐。好的(float4, int, double)能接近cudaMemcpy,差的只能达到二分之一(char) 或者(flaot3)左右。

现在回到CUDA文档里合并访问的要求,如下:
\begin{itemize}
\item 对于每16个线程(半个warp)来说,
\end{itemize}

费了很大的周折,我们终于实现了一个无论数据类型,无论地址偏移的高性能的拷贝,归根结底还是要回到用C语言的实现。